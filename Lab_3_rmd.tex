% Options for packages loaded elsewhere
\PassOptionsToPackage{unicode}{hyperref}
\PassOptionsToPackage{hyphens}{url}
%
\documentclass[
]{article}
\usepackage{amsmath,amssymb}
\usepackage{iftex}
\ifPDFTeX
  \usepackage[T1]{fontenc}
  \usepackage[utf8]{inputenc}
  \usepackage{textcomp} % provide euro and other symbols
\else % if luatex or xetex
  \usepackage{unicode-math} % this also loads fontspec
  \defaultfontfeatures{Scale=MatchLowercase}
  \defaultfontfeatures[\rmfamily]{Ligatures=TeX,Scale=1}
\fi
\usepackage{lmodern}
\ifPDFTeX\else
  % xetex/luatex font selection
\fi
% Use upquote if available, for straight quotes in verbatim environments
\IfFileExists{upquote.sty}{\usepackage{upquote}}{}
\IfFileExists{microtype.sty}{% use microtype if available
  \usepackage[]{microtype}
  \UseMicrotypeSet[protrusion]{basicmath} % disable protrusion for tt fonts
}{}
\makeatletter
\@ifundefined{KOMAClassName}{% if non-KOMA class
  \IfFileExists{parskip.sty}{%
    \usepackage{parskip}
  }{% else
    \setlength{\parindent}{0pt}
    \setlength{\parskip}{6pt plus 2pt minus 1pt}}
}{% if KOMA class
  \KOMAoptions{parskip=half}}
\makeatother
\usepackage{xcolor}
\usepackage[margin=1in]{geometry}
\usepackage{color}
\usepackage{fancyvrb}
\newcommand{\VerbBar}{|}
\newcommand{\VERB}{\Verb[commandchars=\\\{\}]}
\DefineVerbatimEnvironment{Highlighting}{Verbatim}{commandchars=\\\{\}}
% Add ',fontsize=\small' for more characters per line
\usepackage{framed}
\definecolor{shadecolor}{RGB}{248,248,248}
\newenvironment{Shaded}{\begin{snugshade}}{\end{snugshade}}
\newcommand{\AlertTok}[1]{\textcolor[rgb]{0.94,0.16,0.16}{#1}}
\newcommand{\AnnotationTok}[1]{\textcolor[rgb]{0.56,0.35,0.01}{\textbf{\textit{#1}}}}
\newcommand{\AttributeTok}[1]{\textcolor[rgb]{0.13,0.29,0.53}{#1}}
\newcommand{\BaseNTok}[1]{\textcolor[rgb]{0.00,0.00,0.81}{#1}}
\newcommand{\BuiltInTok}[1]{#1}
\newcommand{\CharTok}[1]{\textcolor[rgb]{0.31,0.60,0.02}{#1}}
\newcommand{\CommentTok}[1]{\textcolor[rgb]{0.56,0.35,0.01}{\textit{#1}}}
\newcommand{\CommentVarTok}[1]{\textcolor[rgb]{0.56,0.35,0.01}{\textbf{\textit{#1}}}}
\newcommand{\ConstantTok}[1]{\textcolor[rgb]{0.56,0.35,0.01}{#1}}
\newcommand{\ControlFlowTok}[1]{\textcolor[rgb]{0.13,0.29,0.53}{\textbf{#1}}}
\newcommand{\DataTypeTok}[1]{\textcolor[rgb]{0.13,0.29,0.53}{#1}}
\newcommand{\DecValTok}[1]{\textcolor[rgb]{0.00,0.00,0.81}{#1}}
\newcommand{\DocumentationTok}[1]{\textcolor[rgb]{0.56,0.35,0.01}{\textbf{\textit{#1}}}}
\newcommand{\ErrorTok}[1]{\textcolor[rgb]{0.64,0.00,0.00}{\textbf{#1}}}
\newcommand{\ExtensionTok}[1]{#1}
\newcommand{\FloatTok}[1]{\textcolor[rgb]{0.00,0.00,0.81}{#1}}
\newcommand{\FunctionTok}[1]{\textcolor[rgb]{0.13,0.29,0.53}{\textbf{#1}}}
\newcommand{\ImportTok}[1]{#1}
\newcommand{\InformationTok}[1]{\textcolor[rgb]{0.56,0.35,0.01}{\textbf{\textit{#1}}}}
\newcommand{\KeywordTok}[1]{\textcolor[rgb]{0.13,0.29,0.53}{\textbf{#1}}}
\newcommand{\NormalTok}[1]{#1}
\newcommand{\OperatorTok}[1]{\textcolor[rgb]{0.81,0.36,0.00}{\textbf{#1}}}
\newcommand{\OtherTok}[1]{\textcolor[rgb]{0.56,0.35,0.01}{#1}}
\newcommand{\PreprocessorTok}[1]{\textcolor[rgb]{0.56,0.35,0.01}{\textit{#1}}}
\newcommand{\RegionMarkerTok}[1]{#1}
\newcommand{\SpecialCharTok}[1]{\textcolor[rgb]{0.81,0.36,0.00}{\textbf{#1}}}
\newcommand{\SpecialStringTok}[1]{\textcolor[rgb]{0.31,0.60,0.02}{#1}}
\newcommand{\StringTok}[1]{\textcolor[rgb]{0.31,0.60,0.02}{#1}}
\newcommand{\VariableTok}[1]{\textcolor[rgb]{0.00,0.00,0.00}{#1}}
\newcommand{\VerbatimStringTok}[1]{\textcolor[rgb]{0.31,0.60,0.02}{#1}}
\newcommand{\WarningTok}[1]{\textcolor[rgb]{0.56,0.35,0.01}{\textbf{\textit{#1}}}}
\usepackage{graphicx}
\makeatletter
\def\maxwidth{\ifdim\Gin@nat@width>\linewidth\linewidth\else\Gin@nat@width\fi}
\def\maxheight{\ifdim\Gin@nat@height>\textheight\textheight\else\Gin@nat@height\fi}
\makeatother
% Scale images if necessary, so that they will not overflow the page
% margins by default, and it is still possible to overwrite the defaults
% using explicit options in \includegraphics[width, height, ...]{}
\setkeys{Gin}{width=\maxwidth,height=\maxheight,keepaspectratio}
% Set default figure placement to htbp
\makeatletter
\def\fps@figure{htbp}
\makeatother
\setlength{\emergencystretch}{3em} % prevent overfull lines
\providecommand{\tightlist}{%
  \setlength{\itemsep}{0pt}\setlength{\parskip}{0pt}}
\setcounter{secnumdepth}{-\maxdimen} % remove section numbering
\ifLuaTeX
  \usepackage{selnolig}  % disable illegal ligatures
\fi
\IfFileExists{bookmark.sty}{\usepackage{bookmark}}{\usepackage{hyperref}}
\IfFileExists{xurl.sty}{\usepackage{xurl}}{} % add URL line breaks if available
\urlstyle{same}
\hypersetup{
  pdftitle={Lab 3},
  pdfauthor={Lily Heidger},
  hidelinks,
  pdfcreator={LaTeX via pandoc}}

\title{Lab 3}
\author{Lily Heidger}
\date{2024-02-01}

\begin{document}
\maketitle

\begin{Shaded}
\begin{Highlighting}[]
\FunctionTok{library}\NormalTok{(dplyr)}
\end{Highlighting}
\end{Shaded}

\begin{verbatim}
## 
## Attaching package: 'dplyr'
\end{verbatim}

\begin{verbatim}
## The following objects are masked from 'package:stats':
## 
##     filter, lag
\end{verbatim}

\begin{verbatim}
## The following objects are masked from 'package:base':
## 
##     intersect, setdiff, setequal, union
\end{verbatim}

\begin{Shaded}
\begin{Highlighting}[]
\FunctionTok{library}\NormalTok{(tidyverse)}
\end{Highlighting}
\end{Shaded}

\begin{verbatim}
## -- Attaching core tidyverse packages ------------------------ tidyverse 2.0.0 --
## v forcats   1.0.0     v readr     2.1.5
## v ggplot2   3.4.4     v stringr   1.5.1
## v lubridate 1.9.3     v tibble    3.2.1
## v purrr     1.0.2     v tidyr     1.3.0
\end{verbatim}

\begin{verbatim}
## -- Conflicts ------------------------------------------ tidyverse_conflicts() --
## x dplyr::filter() masks stats::filter()
## x dplyr::lag()    masks stats::lag()
## i Use the conflicted package (<http://conflicted.r-lib.org/>) to force all conflicts to become errors
\end{verbatim}

\begin{Shaded}
\begin{Highlighting}[]
\FunctionTok{library}\NormalTok{(here)}
\end{Highlighting}
\end{Shaded}

\begin{verbatim}
## here() starts at C:/Users/lily/Documents/210B/210B Lab 3
\end{verbatim}

\begin{Shaded}
\begin{Highlighting}[]
\FunctionTok{library}\NormalTok{(patchwork)}
\end{Highlighting}
\end{Shaded}

pnorm = find probabilities associated with values in the distribution
qnorm = find values associated with probabilities in the distribution

\hypertarget{snowfall-for-a-location-is-found-to-be-normally-distributed-with-mean-96-inches-and-standard-deviation-32-inches.}{%
\paragraph{1. Snowfall for a location is found to be normally
distributed with mean 96 inches and standard deviation 32
inches.}\label{snowfall-for-a-location-is-found-to-be-normally-distributed-with-mean-96-inches-and-standard-deviation-32-inches.}}

\hypertarget{a-what-is-the-probability-that-a-given-year-will-have-more-than-120-inches-of-snow}{%
\subparagraph{a) What is the probability that a given year will have
more than 120 inches of
snow?}\label{a-what-is-the-probability-that-a-given-year-will-have-more-than-120-inches-of-snow}}

The probability any given year will have more than 120 inches of snow is
0.227 or \textasciitilde23\%.

\begin{Shaded}
\begin{Highlighting}[]
\NormalTok{z\_score }\OtherTok{\textless{}{-}}\NormalTok{ (}\DecValTok{120{-}96}\NormalTok{)}\SpecialCharTok{/}\DecValTok{32}

\NormalTok{probability }\OtherTok{\textless{}{-}} \FunctionTok{pnorm}\NormalTok{(z\_score)}

\CommentTok{\# Print the result}
\FunctionTok{print}\NormalTok{(}\DecValTok{1}\SpecialCharTok{{-}}\NormalTok{probability)}
\end{Highlighting}
\end{Shaded}

\begin{verbatim}
## [1] 0.2266274
\end{verbatim}

\hypertarget{b-what-is-the-probability-that-the-snowfall-will-be-between-90-and-100-inches}{%
\subparagraph{b) What is the probability that the snowfall will be
between 90 and 100
inches?}\label{b-what-is-the-probability-that-the-snowfall-will-be-between-90-and-100-inches}}

The probability that the snowfall will be between 90 and 100 inches is
0.1241039 or \textasciitilde12\%.

\begin{Shaded}
\begin{Highlighting}[]
\NormalTok{hundred\_z\_score }\OtherTok{\textless{}{-}}\NormalTok{ (}\DecValTok{100{-}96}\NormalTok{)}\SpecialCharTok{/}\DecValTok{32}

\NormalTok{hundred\_probability }\OtherTok{\textless{}{-}} \FunctionTok{pnorm}\NormalTok{(hundred\_z\_score)}

\FunctionTok{print}\NormalTok{(hundred\_probability)}
\end{Highlighting}
\end{Shaded}

\begin{verbatim}
## [1] 0.5497382
\end{verbatim}

\begin{Shaded}
\begin{Highlighting}[]
\CommentTok{\# 0.5497382}


\NormalTok{ninety\_z\_score }\OtherTok{\textless{}{-}}\NormalTok{ (}\DecValTok{90{-}96}\NormalTok{)}\SpecialCharTok{/}\DecValTok{32}

\NormalTok{ninety\_probability }\OtherTok{\textless{}{-}} \FunctionTok{pnorm}\NormalTok{(ninety\_z\_score)}

\FunctionTok{print}\NormalTok{(ninety\_probability)}
\end{Highlighting}
\end{Shaded}

\begin{verbatim}
## [1] 0.4256343
\end{verbatim}

\begin{Shaded}
\begin{Highlighting}[]
\CommentTok{\# 0.4256343}

\NormalTok{hundred\_probability}\SpecialCharTok{{-}}\NormalTok{ ninety\_probability}
\end{Highlighting}
\end{Shaded}

\begin{verbatim}
## [1] 0.1241039
\end{verbatim}

\hypertarget{c-what-level-of-snowfall-will-be-exceeded-only-10-of-the-time}{%
\subparagraph{c) What level of snowfall will be exceeded only 10\% of
the
time?}\label{c-what-level-of-snowfall-will-be-exceeded-only-10-of-the-time}}

An amount of snow over 137 inches will only be exceeded 10\% of the
time.

\begin{Shaded}
\begin{Highlighting}[]
\NormalTok{mean\_snowfall }\OtherTok{\textless{}{-}} \DecValTok{96}
\NormalTok{sd\_snowfall }\OtherTok{\textless{}{-}} \DecValTok{32}
\NormalTok{probability\_exceeded }\OtherTok{\textless{}{-}} \FloatTok{0.10}

\CommentTok{\#qnorm calculates quantile based on normal distribution}
\NormalTok{level\_of\_snowfall }\OtherTok{\textless{}{-}} \FunctionTok{qnorm}\NormalTok{(}\DecValTok{1} \SpecialCharTok{{-}}\NormalTok{ probability\_exceeded, mean\_snowfall, sd\_snowfall)}

\NormalTok{level\_of\_snowfall}
\end{Highlighting}
\end{Shaded}

\begin{verbatim}
## [1] 137.0097
\end{verbatim}

\hypertarget{assume-that-the-prices-paid-for-housing-within-a-neighborhood-have-a-normal-distribution-with-mean-100000-and-standard-deviation-35000.}{%
\paragraph{2. Assume that the prices paid for housing within a
neighborhood have a normal distribution, with mean \$100,000, and
standard deviation
\$35,000.}\label{assume-that-the-prices-paid-for-housing-within-a-neighborhood-have-a-normal-distribution-with-mean-100000-and-standard-deviation-35000.}}

\hypertarget{a-what-percentage-of-houses-in-the-neighborhood-have-prices-between-90000-and-130000}{%
\subparagraph{a) What percentage of houses in the neighborhood have
prices between \$90,000 and
\$130,000?}\label{a-what-percentage-of-houses-in-the-neighborhood-have-prices-between-90000-and-130000}}

Approximately 42\% of houses have prices between \$90,000 and \$130,000.

\begin{Shaded}
\begin{Highlighting}[]
\NormalTok{one\_thirty\_z }\OtherTok{\textless{}{-}}\NormalTok{ (}\DecValTok{130000{-}100000}\NormalTok{)}\SpecialCharTok{/}\DecValTok{35000}

\NormalTok{one\_thirty\_probability }\OtherTok{\textless{}{-}} \FunctionTok{pnorm}\NormalTok{(one\_thirty\_z)}

\FunctionTok{print}\NormalTok{(one\_thirty\_probability)}
\end{Highlighting}
\end{Shaded}

\begin{verbatim}
## [1] 0.804317
\end{verbatim}

\begin{Shaded}
\begin{Highlighting}[]
\CommentTok{\#0.804317}

\NormalTok{ninety\_thou\_z }\OtherTok{\textless{}{-}}\NormalTok{ (}\DecValTok{90000{-}100000}\NormalTok{)}\SpecialCharTok{/}\DecValTok{35000}

\NormalTok{ninety\_thou\_probability }\OtherTok{\textless{}{-}} \FunctionTok{pnorm}\NormalTok{(ninety\_thou\_z)}

\FunctionTok{print}\NormalTok{(ninety\_thou\_probability)}
\end{Highlighting}
\end{Shaded}

\begin{verbatim}
## [1] 0.3875485
\end{verbatim}

\begin{Shaded}
\begin{Highlighting}[]
\CommentTok{\#0.3875485}

\NormalTok{one\_thirty\_probability}\SpecialCharTok{{-}}\NormalTok{ ninety\_thou\_probability}
\end{Highlighting}
\end{Shaded}

\begin{verbatim}
## [1] 0.4167685
\end{verbatim}

\hypertarget{b-what-price-of-housing-is-such-that-only-12-of-all-houses-in-the-neighborhood-have-lower-prices}{%
\subparagraph{b) What price of housing is such that only 12\% of all
houses in the neighborhood have lower
prices?}\label{b-what-price-of-housing-is-such-that-only-12-of-all-houses-in-the-neighborhood-have-lower-prices}}

Only 12\% of all houses in the neighborhood cost less than \$58,875.46.

\begin{Shaded}
\begin{Highlighting}[]
\NormalTok{mean\_price }\OtherTok{\textless{}{-}} \DecValTok{100000}
\NormalTok{sd\_price }\OtherTok{\textless{}{-}} \DecValTok{35000}
\NormalTok{probability\_lower }\OtherTok{\textless{}{-}} \FloatTok{0.12}

\NormalTok{housing\_price }\OtherTok{\textless{}{-}} \FunctionTok{qnorm}\NormalTok{(probability\_lower, mean\_price, sd\_price)}

\NormalTok{housing\_price}
\end{Highlighting}
\end{Shaded}

\begin{verbatim}
## [1] 58875.46
\end{verbatim}

\hypertarget{residents-in-a-community-have-a-choice-of-six-different-grocery-stores.-the-proportions-of-residents-observed-to-patronize-each-are-p1-0.4-p2-0.25-p3-0.15-p4-0.1-p5-0.05-and-p6-0.05-where-the-stores-are-arranged-in-terms-of-increasing-distance-from-the-residential-community.-fit-an-intervening-opportunities-model-to-these-data-by-estimating-the-parameter-l.}{%
\paragraph{3. Residents in a community have a choice of six different
grocery stores. The proportions of residents observed to patronize each
are p(1) = 0.4, p(2) = 0.25, p(3) = 0.15, p(4) = 0.1, p(5) = 0.05, and
p(6) = 0.05, where the stores are arranged in terms of increasing
distance from the residential community. Fit an intervening
opportunities model to these data by estimating the parameter
L.}\label{residents-in-a-community-have-a-choice-of-six-different-grocery-stores.-the-proportions-of-residents-observed-to-patronize-each-are-p1-0.4-p2-0.25-p3-0.15-p4-0.1-p5-0.05-and-p6-0.05-where-the-stores-are-arranged-in-terms-of-increasing-distance-from-the-residential-community.-fit-an-intervening-opportunities-model-to-these-data-by-estimating-the-parameter-l.}}

\begin{Shaded}
\begin{Highlighting}[]
\NormalTok{mean }\OtherTok{\textless{}{-}} \FloatTok{2.2}
\NormalTok{L }\OtherTok{\textless{}{-}} \DecValTok{1}\SpecialCharTok{/}\FloatTok{2.2}

\NormalTok{df1}\OtherTok{\textless{}{-}} \FunctionTok{data.frame}\NormalTok{(}\AttributeTok{values =} \DecValTok{1}\SpecialCharTok{:}\DecValTok{6}\NormalTok{,}
\AttributeTok{prop =} \FunctionTok{c}\NormalTok{(}\FloatTok{0.4}\NormalTok{, }\FloatTok{0.25}\NormalTok{, }\FloatTok{0.15}\NormalTok{, }\FloatTok{0.1}\NormalTok{, }\FloatTok{0.05}\NormalTok{, }\FloatTok{0.05}\NormalTok{),}
\AttributeTok{prob =} \ConstantTok{NA}\NormalTok{)}

\NormalTok{prob\_function }\OtherTok{\textless{}{-}} \ControlFlowTok{function}\NormalTok{(L, n) \{}
\NormalTok{numerator }\OtherTok{\textless{}{-}}\NormalTok{ (}\DecValTok{1} \SpecialCharTok{{-}}\NormalTok{ L)}\SpecialCharTok{\^{}}\NormalTok{(}\DecValTok{1}\SpecialCharTok{:}\NormalTok{n }\SpecialCharTok{{-}} \DecValTok{1}\NormalTok{) }\SpecialCharTok{*}\NormalTok{ L}
\NormalTok{denominator }\OtherTok{\textless{}{-}} \FunctionTok{sum}\NormalTok{((}\DecValTok{1} \SpecialCharTok{{-}}\NormalTok{ L)}\SpecialCharTok{\^{}}\NormalTok{(}\DecValTok{1}\SpecialCharTok{:}\NormalTok{n }\SpecialCharTok{{-}} \DecValTok{1}\NormalTok{) }\SpecialCharTok{*}\NormalTok{ L)}
\NormalTok{result }\OtherTok{\textless{}{-}}\NormalTok{ numerator }\SpecialCharTok{/}\NormalTok{ denominator}
\FunctionTok{return}\NormalTok{(result)}
\NormalTok{\}}

\NormalTok{L }\OtherTok{\textless{}{-}} \DecValTok{1} \SpecialCharTok{/} \FunctionTok{sum}\NormalTok{(df1}\SpecialCharTok{$}\NormalTok{values }\SpecialCharTok{*}\NormalTok{ df1}\SpecialCharTok{$}\NormalTok{prop)}
\NormalTok{n }\OtherTok{\textless{}{-}} \DecValTok{6}
\NormalTok{result }\OtherTok{\textless{}{-}} \FunctionTok{prob\_function}\NormalTok{(L, n)}
\NormalTok{df1}\OtherTok{\textless{}{-}}\NormalTok{ df1 }\SpecialCharTok{\%\textgreater{}\%}
\FunctionTok{mutate}\NormalTok{(}\AttributeTok{prob =}\NormalTok{ (result))}

\NormalTok{df1}
\end{Highlighting}
\end{Shaded}

\begin{verbatim}
##   values prop       prob
## 1      1 0.40 0.44943680
## 2      2 0.25 0.25402949
## 3      3 0.15 0.14358189
## 4      4 0.10 0.08115498
## 5      5 0.05 0.04587021
## 6      6 0.05 0.02592664
\end{verbatim}

\begin{Shaded}
\begin{Highlighting}[]
\FunctionTok{ggplot}\NormalTok{(df1, }\FunctionTok{aes}\NormalTok{(}\AttributeTok{x =}\NormalTok{ values)) }\SpecialCharTok{+}
  \FunctionTok{geom\_line}\NormalTok{(}\FunctionTok{aes}\NormalTok{(}\AttributeTok{y =}\NormalTok{ prop, }\AttributeTok{color =} \StringTok{"prop"}\NormalTok{), }\AttributeTok{size =} \DecValTok{1}\NormalTok{) }\SpecialCharTok{+}
  \FunctionTok{geom\_line}\NormalTok{(}\FunctionTok{aes}\NormalTok{(}\AttributeTok{y =}\NormalTok{ prob, }\AttributeTok{color =} \StringTok{"prob"}\NormalTok{), }\AttributeTok{size =} \DecValTok{1}\NormalTok{) }\SpecialCharTok{+}
  \FunctionTok{labs}\NormalTok{(}\AttributeTok{title =} \StringTok{"Intervening Opp. Model"}\NormalTok{,}
       \AttributeTok{color =} \StringTok{"Legend"}\NormalTok{) }\SpecialCharTok{+}
  \FunctionTok{scale\_color\_manual}\NormalTok{(}\AttributeTok{values =} \FunctionTok{c}\NormalTok{(}\StringTok{"prop"} \OtherTok{=} \StringTok{"blue"}\NormalTok{, }\StringTok{"prob"} \OtherTok{=} \StringTok{"red"}\NormalTok{)) }\SpecialCharTok{+}
  \FunctionTok{theme\_minimal}\NormalTok{()}
\end{Highlighting}
\end{Shaded}

\begin{verbatim}
## Warning: Using `size` aesthetic for lines was deprecated in ggplot2 3.4.0.
## i Please use `linewidth` instead.
## This warning is displayed once every 8 hours.
## Call `lifecycle::last_lifecycle_warnings()` to see where this warning was
## generated.
\end{verbatim}

\includegraphics{Lab_3_rmd_files/figure-latex/unnamed-chunk-6-1.pdf}

\hypertarget{the-annual-probability-that-suburban-residents-move-to-the-central-city-is-0.08-while-the-annual-probability-that-the-city-residents-move-to-the-suburbs-is-0.11.-starting-with-respective-populations-of-30000-and-20000-in-the-central-city-and-suburbs-respectively-forecast-the-population-redistribution-that-will-occur-over-the-next-three-years.-use-the-markov-model-assumption-that-the-probabilities-of-movement-will-remain-constant.-also-find-the-long-run-equilibrium-populations.}{%
\paragraph{4. The annual probability that suburban residents move to the
central city is 0.08, while the annual probability that the city
residents move to the suburbs is 0.11. Starting with respective
populations of 30,000 and 20,000 in the central city and suburbs,
respectively, forecast the population redistribution that will occur
over the next three years. Use the Markov model assumption that the
probabilities of movement will remain constant. Also find the long-run,
equilibrium
populations.}\label{the-annual-probability-that-suburban-residents-move-to-the-central-city-is-0.08-while-the-annual-probability-that-the-city-residents-move-to-the-suburbs-is-0.11.-starting-with-respective-populations-of-30000-and-20000-in-the-central-city-and-suburbs-respectively-forecast-the-population-redistribution-that-will-occur-over-the-next-three-years.-use-the-markov-model-assumption-that-the-probabilities-of-movement-will-remain-constant.-also-find-the-long-run-equilibrium-populations.}}

The population forecasts for the next three years are as follows: Year
1: City: 28300 Suburbs: 21700

Year 2: City: 21577 Suburbs: 28423

Year 3: City: 21477 Suburbs: 28523

City Equilibrium = 21053 Suburbs Equilibrium = 28945

\begin{Shaded}
\begin{Highlighting}[]
\NormalTok{total\_pop }\OtherTok{\textless{}{-}} \DecValTok{50000}
\NormalTok{Pcity1 }\OtherTok{\textless{}{-}}\NormalTok{ (}\FloatTok{0.89}\SpecialCharTok{*}\DecValTok{30000}\NormalTok{)}\SpecialCharTok{+}\NormalTok{(}\FloatTok{0.08}\SpecialCharTok{*}\DecValTok{20000}\NormalTok{)}
\NormalTok{Psub1 }\OtherTok{\textless{}{-}}\NormalTok{ (}\FloatTok{0.92}\SpecialCharTok{*}\DecValTok{20000}\NormalTok{)}\SpecialCharTok{+}\NormalTok{(}\FloatTok{0.11}\SpecialCharTok{*}\DecValTok{30000}\NormalTok{)}

\DecValTok{50000}\SpecialCharTok{{-}}\NormalTok{Psub1}
\end{Highlighting}
\end{Shaded}

\begin{verbatim}
## [1] 28300
\end{verbatim}

\begin{Shaded}
\begin{Highlighting}[]
\CommentTok{\#year 2}
\NormalTok{Psub2 }\OtherTok{\textless{}{-}}\NormalTok{  (}\FloatTok{0.92}\SpecialCharTok{*}\DecValTok{28300}\NormalTok{)}\SpecialCharTok{+}\NormalTok{(}\FloatTok{0.11}\SpecialCharTok{*}\DecValTok{21700}\NormalTok{)}

\DecValTok{50000}\SpecialCharTok{{-}}\NormalTok{ Psub2}
\end{Highlighting}
\end{Shaded}

\begin{verbatim}
## [1] 21577
\end{verbatim}

\begin{Shaded}
\begin{Highlighting}[]
\CommentTok{\#year 3}
\NormalTok{Psub3 }\OtherTok{\textless{}{-}}\NormalTok{ (}\FloatTok{0.92} \SpecialCharTok{*}\DecValTok{28423}\NormalTok{) }\SpecialCharTok{+}\NormalTok{ (}\FloatTok{0.11}\SpecialCharTok{*}\DecValTok{21577}\NormalTok{)}
\NormalTok{Psub3}
\end{Highlighting}
\end{Shaded}

\begin{verbatim}
## [1] 28522.63
\end{verbatim}

\begin{Shaded}
\begin{Highlighting}[]
\DecValTok{50000}\SpecialCharTok{{-}}\NormalTok{Psub3}
\end{Highlighting}
\end{Shaded}

\begin{verbatim}
## [1] 21477.37
\end{verbatim}

\begin{Shaded}
\begin{Highlighting}[]
\CommentTok{\#PcityEquil \textless{}{-} (0.89*x) + (0.08 * y)}
\CommentTok{\#PcityEquil}

\CommentTok{\# Pcity = 0.89(Pcity) +0.08(50,000 {-} Pcity)}

\NormalTok{PcityEquil }\OtherTok{\textless{}{-}} \FloatTok{21052.63}
\DecValTok{50000}\SpecialCharTok{{-}}\NormalTok{ PcityEquil}
\end{Highlighting}
\end{Shaded}

\begin{verbatim}
## [1] 28947.37
\end{verbatim}

\hypertarget{the-magnitude-richter-scale-of-earthquakes-along-a-californian-fault-is-exponentially-distributed-with-l-12.35.}{%
\paragraph{5. The magnitude (Richter scale) of earthquakes along a
Californian fault is exponentially distributed, with l =
(1/2.35).}\label{the-magnitude-richter-scale-of-earthquakes-along-a-californian-fault-is-exponentially-distributed-with-l-12.35.}}

\hypertarget{what-is-the-probability-of-an-earthquake-exceeding-magnitude-6.3}{%
\subparagraph{What is the probability of an earthquake exceeding
magnitude
6.3?}\label{what-is-the-probability-of-an-earthquake-exceeding-magnitude-6.3}}

The probability of an earthquake exceeding magnitude 6.3 is 0.069.

\begin{Shaded}
\begin{Highlighting}[]
\DecValTok{1}\SpecialCharTok{{-}}\NormalTok{ (}\DecValTok{1}\SpecialCharTok{{-}} \FunctionTok{exp}\NormalTok{(}\SpecialCharTok{{-}}\FloatTok{6.3} \SpecialCharTok{/} \FloatTok{2.35}\NormalTok{))}
\end{Highlighting}
\end{Shaded}

\begin{verbatim}
## [1] 0.06850483
\end{verbatim}

\hypertarget{what-is-the-probability-of-an-earthquake-during-the-year-that-exceeds-magnitude-7.7}{%
\subparagraph{What is the probability of an earthquake during the year
that exceeds magnitude
7.7?}\label{what-is-the-probability-of-an-earthquake-during-the-year-that-exceeds-magnitude-7.7}}

The probability that an earthquake during the year exceeds magnitude 7.7
is 0.038.

\begin{Shaded}
\begin{Highlighting}[]
\DecValTok{1}\SpecialCharTok{{-}}\NormalTok{ (}\DecValTok{1} \SpecialCharTok{{-}} \FunctionTok{exp}\NormalTok{(}\SpecialCharTok{{-}}\FloatTok{7.7} \SpecialCharTok{/} \FloatTok{2.35}\NormalTok{))}
\end{Highlighting}
\end{Shaded}

\begin{verbatim}
## [1] 0.03775657
\end{verbatim}

\hypertarget{a-variable-x-is-uniformly-distributed-between-10-and-24.}{%
\paragraph{6. A variable, X, is uniformly distributed between 10 and
24.}\label{a-variable-x-is-uniformly-distributed-between-10-and-24.}}

\hypertarget{a-what-is-p16x20}{%
\subparagraph{(a) What is P(16≤x≤20)?}\label{a-what-is-p16x20}}

The probability that x is greater than or equal to 16 and less than or
equal to 20 is 0.286.

\begin{Shaded}
\begin{Highlighting}[]
\NormalTok{(}\DecValTok{20} \SpecialCharTok{{-}} \DecValTok{16}\NormalTok{)}\SpecialCharTok{/}\NormalTok{(}\DecValTok{24} \SpecialCharTok{{-}} \DecValTok{10}\NormalTok{)}
\end{Highlighting}
\end{Shaded}

\begin{verbatim}
## [1] 0.2857143
\end{verbatim}

\hypertarget{b-what-is-the-mean-and-variance-of-x}{%
\subparagraph{(b) What is the mean and variance of
X?}\label{b-what-is-the-mean-and-variance-of-x}}

The mean is 17, and the variance is 16.33.

\begin{Shaded}
\begin{Highlighting}[]
\NormalTok{mean }\OtherTok{\textless{}{-}}\NormalTok{ (}\DecValTok{10} \SpecialCharTok{+} \DecValTok{24}\NormalTok{)}\SpecialCharTok{/}\DecValTok{2}
\NormalTok{var }\OtherTok{\textless{}{-}}\NormalTok{ ((}\DecValTok{24{-}10}\NormalTok{)}\SpecialCharTok{\^{}}\DecValTok{2}\NormalTok{)}\SpecialCharTok{/}\DecValTok{12}

\NormalTok{mean}
\end{Highlighting}
\end{Shaded}

\begin{verbatim}
## [1] 17
\end{verbatim}

\begin{Shaded}
\begin{Highlighting}[]
\NormalTok{var}
\end{Highlighting}
\end{Shaded}

\begin{verbatim}
## [1] 16.33333
\end{verbatim}

\hypertarget{the-duration-of-residences-in-households-is-found-to-be-exponentially-distributed-with-mean-4.76-years.}{%
\paragraph{7. The duration of residences in households is found to be
exponentially distributed with mean 4.76
years.}\label{the-duration-of-residences-in-households-is-found-to-be-exponentially-distributed-with-mean-4.76-years.}}

\begin{Shaded}
\begin{Highlighting}[]
\NormalTok{mean\_duration }\OtherTok{\textless{}{-}} \FloatTok{4.76}
\end{Highlighting}
\end{Shaded}

\hypertarget{what-is-the-probability-that-a-family-is-in-their-house-for-more-than-8-years}{%
\subparagraph{What is the probability that a family is in their house
for more than 8
years?}\label{what-is-the-probability-that-a-family-is-in-their-house-for-more-than-8-years}}

The probability that a family is in their house for more than 8 years is
0.18.

\begin{Shaded}
\begin{Highlighting}[]
\NormalTok{years }\OtherTok{\textless{}{-}} \DecValTok{8}

\NormalTok{prob\_more\_than\_8 }\OtherTok{\textless{}{-}} \DecValTok{1} \SpecialCharTok{{-}} \FunctionTok{pexp}\NormalTok{(years, }\AttributeTok{rate =} \DecValTok{1} \SpecialCharTok{/}\NormalTok{ mean\_duration)}

\NormalTok{prob\_more\_than\_8}
\end{Highlighting}
\end{Shaded}

\begin{verbatim}
## [1] 0.1862487
\end{verbatim}

\hypertarget{between-5-and-8-years}{%
\subparagraph{Between 5 and 8 years?}\label{between-5-and-8-years}}

The probability that a family is in their house for more between 5 and 8
years is 0.162.

\begin{Shaded}
\begin{Highlighting}[]
\NormalTok{lower\_bound }\OtherTok{\textless{}{-}} \DecValTok{5}
\NormalTok{upper\_bound }\OtherTok{\textless{}{-}} \DecValTok{8}

\NormalTok{prob\_lower }\OtherTok{\textless{}{-}} \FunctionTok{pexp}\NormalTok{(lower\_bound, }\AttributeTok{rate =} \DecValTok{1}\SpecialCharTok{/}\NormalTok{ mean\_duration)}
\NormalTok{prob\_upper }\OtherTok{\textless{}{-}} \FunctionTok{pexp}\NormalTok{(upper\_bound, }\AttributeTok{rate =} \DecValTok{1}\SpecialCharTok{/}\NormalTok{mean\_duration)}

\NormalTok{prob\_upper }\SpecialCharTok{{-}}\NormalTok{ prob\_lower}
\end{Highlighting}
\end{Shaded}

\begin{verbatim}
## [1] 0.163542
\end{verbatim}

\hypertarget{the-mean-value-of-annual-imports-for-a-country-is-normally-distributed-with-mean-30-million-and-standard-deviation-16-million.}{%
\paragraph{8. The mean value of annual imports for a country is normally
distributed with mean \$30 million and standard deviation \$16
million.}\label{the-mean-value-of-annual-imports-for-a-country-is-normally-distributed-with-mean-30-million-and-standard-deviation-16-million.}}

\begin{Shaded}
\begin{Highlighting}[]
\NormalTok{mean\_imports }\OtherTok{\textless{}{-}} \DecValTok{30}
\NormalTok{sd\_imports }\OtherTok{\textless{}{-}} \DecValTok{16}
\end{Highlighting}
\end{Shaded}

\hypertarget{what-dollar-value-of-imports-is-exceeded-only-5-of-the-time}{%
\subparagraph{What dollar value of imports is exceeded only 5\% of the
time?}\label{what-dollar-value-of-imports-is-exceeded-only-5-of-the-time}}

A dollar value of \$56.32 Million is only exceeded 5\% of the time.

\begin{Shaded}
\begin{Highlighting}[]
\NormalTok{prob\_5\_percent }\OtherTok{\textless{}{-}} \FloatTok{0.05}

\NormalTok{imports }\OtherTok{\textless{}{-}} \FunctionTok{qnorm}\NormalTok{(}\DecValTok{1} \SpecialCharTok{{-}}\NormalTok{ prob\_5\_percent, mean\_imports, sd\_imports)}

\NormalTok{imports}
\end{Highlighting}
\end{Shaded}

\begin{verbatim}
## [1] 56.31766
\end{verbatim}

\hypertarget{what-fraction-of-years-have-import-values-between-29-and-45-million}{%
\subparagraph{What fraction of years have import values between 29 and
45
million?}\label{what-fraction-of-years-have-import-values-between-29-and-45-million}}

0.351 of years have import values between 29 and 45 million.

\begin{Shaded}
\begin{Highlighting}[]
\NormalTok{lower\_bound }\OtherTok{\textless{}{-}} \DecValTok{29}
\NormalTok{upper\_bound }\OtherTok{\textless{}{-}} \DecValTok{45}

\NormalTok{probability\_lower }\OtherTok{\textless{}{-}} \FunctionTok{pnorm}\NormalTok{(lower\_bound, mean\_imports, sd\_imports)}
\NormalTok{probability\_upper }\OtherTok{\textless{}{-}} \FunctionTok{pnorm}\NormalTok{(upper\_bound, mean\_imports, sd\_imports)}

\NormalTok{fraction }\OtherTok{\textless{}{-}}\NormalTok{ probability\_upper }\SpecialCharTok{{-}}\NormalTok{ probability\_lower}

\NormalTok{fraction}
\end{Highlighting}
\end{Shaded}

\begin{verbatim}
## [1] 0.350667
\end{verbatim}

\hypertarget{the-number-of-customers-at-a-bank-each-day-is-found-to-be-normally-distributed-with-mean-250-and-standard-deviation-of-110.}{%
\paragraph{9. The number of customers at a bank each day is found to be
normally distributed with mean 250 and standard deviation of
110.}\label{the-number-of-customers-at-a-bank-each-day-is-found-to-be-normally-distributed-with-mean-250-and-standard-deviation-of-110.}}

\begin{Shaded}
\begin{Highlighting}[]
\NormalTok{mean\_customers }\OtherTok{\textless{}{-}} \DecValTok{250}
\NormalTok{sd\_customers }\OtherTok{\textless{}{-}} \DecValTok{110}
\end{Highlighting}
\end{Shaded}

\hypertarget{what-fraction-of-days-will-have-less-than-100-customers}{%
\subparagraph{What fraction of days will have less than 100
customers?}\label{what-fraction-of-days-will-have-less-than-100-customers}}

0.0863 or \textasciitilde8.6\% of days will have less than 100
customers.

\begin{Shaded}
\begin{Highlighting}[]
\NormalTok{less\_than\_100 }\OtherTok{\textless{}{-}} \DecValTok{100}

\NormalTok{probability\_less\_than\_100 }\OtherTok{\textless{}{-}} \FunctionTok{pnorm}\NormalTok{(less\_than\_100, mean\_customers, sd\_customers)}


\NormalTok{probability\_less\_than\_100}
\end{Highlighting}
\end{Shaded}

\begin{verbatim}
## [1] 0.08634102
\end{verbatim}

\hypertarget{more-than-320}{%
\subparagraph{More than 320?}\label{more-than-320}}

0.2622 or \textasciitilde26\% of days will have more than 320 customers.

\begin{Shaded}
\begin{Highlighting}[]
\NormalTok{more\_than\_320 }\OtherTok{\textless{}{-}} \DecValTok{320}

\NormalTok{probability\_more\_than\_320 }\OtherTok{\textless{}{-}} \FunctionTok{pnorm}\NormalTok{(more\_than\_320, mean\_customers, sd\_customers, }\AttributeTok{lower.tail =} \ConstantTok{FALSE}\NormalTok{) }

\NormalTok{probability\_more\_than\_320}
\end{Highlighting}
\end{Shaded}

\begin{verbatim}
## [1] 0.2622697
\end{verbatim}

\hypertarget{what-number-of-customers-will-be-exceeded-10-of-the-time}{%
\subparagraph{What number of customers will be exceeded 10\% of the
time?}\label{what-number-of-customers-will-be-exceeded-10-of-the-time}}

10\% of the time, the number of customers will exceed 109.

\begin{Shaded}
\begin{Highlighting}[]
\NormalTok{prob\_10\_percent }\OtherTok{\textless{}{-}} \FloatTok{0.10}

\NormalTok{customers\_exceeded\_10\_percent }\OtherTok{\textless{}{-}} \FunctionTok{qnorm}\NormalTok{(prob\_10\_percent, mean\_customers, sd\_customers)}

\NormalTok{customers\_exceeded\_10\_percent}
\end{Highlighting}
\end{Shaded}

\begin{verbatim}
## [1] 109.0293
\end{verbatim}

\hypertarget{incomes-are-exponentially-distributed-with-a-mean-of-10000.-what-fraction-of-the-population-has-income}{%
\paragraph{10. Incomes are exponentially distributed with a mean of
\$10,000. What fraction of the population has
income:}\label{incomes-are-exponentially-distributed-with-a-mean-of-10000.-what-fraction-of-the-population-has-income}}

\hypertarget{a-less-than-or-equal-to-8000}{%
\subparagraph{a) Less than or equal to
\$8000?}\label{a-less-than-or-equal-to-8000}}

About 0.550671 or 55\% of the population has income less than or equal
to \$8000.

\begin{Shaded}
\begin{Highlighting}[]
\NormalTok{mean\_income }\OtherTok{\textless{}{-}} \DecValTok{10000}
\NormalTok{rate }\OtherTok{\textless{}{-}} \DecValTok{1}\SpecialCharTok{/}\NormalTok{mean\_income}

\NormalTok{desired\_income }\OtherTok{\textless{}{-}} \DecValTok{8000}

\CommentTok{\#cdf function for exp model}
\NormalTok{probability\_less\_than\_8000 }\OtherTok{\textless{}{-}} \FunctionTok{pexp}\NormalTok{(desired\_income, rate)}

\NormalTok{probability\_less\_than\_8000}
\end{Highlighting}
\end{Shaded}

\begin{verbatim}
## [1] 0.550671
\end{verbatim}

\hypertarget{b-greater-than-12000}{%
\subparagraph{b) Greater than \$12,000?}\label{b-greater-than-12000}}

About 0.3011 or 30\% of the population has income greater than \$12,000.

\begin{Shaded}
\begin{Highlighting}[]
\NormalTok{income\_greater\_than\_12000 }\OtherTok{\textless{}{-}} \DecValTok{12000}

\CommentTok{\# cdf for exp but only upper tail}
\NormalTok{probability\_greater\_than\_12000 }\OtherTok{\textless{}{-}} \FunctionTok{pexp}\NormalTok{(income\_greater\_than\_12000, rate, }\AttributeTok{lower.tail =} \ConstantTok{FALSE}\NormalTok{)}

\NormalTok{probability\_greater\_than\_12000}
\end{Highlighting}
\end{Shaded}

\begin{verbatim}
## [1] 0.3011942
\end{verbatim}

\hypertarget{c-between-9000-and-12000}{%
\subparagraph{c) Between \$9,000 and
\$12,000?}\label{c-between-9000-and-12000}}

About 0.1053754 or 11\% of the population has income between \$9,000 and
\$12,000.

\begin{Shaded}
\begin{Highlighting}[]
\NormalTok{lower\_bound }\OtherTok{\textless{}{-}} \DecValTok{9000}
\NormalTok{upper\_bound }\OtherTok{\textless{}{-}} \DecValTok{12000}

\NormalTok{probability\_lower }\OtherTok{\textless{}{-}} \FunctionTok{pexp}\NormalTok{(lower\_bound, rate)}
\NormalTok{probability\_upper }\OtherTok{\textless{}{-}} \FunctionTok{pexp}\NormalTok{(upper\_bound, rate)}


\NormalTok{probability\_between\_9000\_and\_12000 }\OtherTok{\textless{}{-}}\NormalTok{ probability\_upper }\SpecialCharTok{{-}}\NormalTok{ probability\_lower}

\NormalTok{probability\_between\_9000\_and\_12000}
\end{Highlighting}
\end{Shaded}

\begin{verbatim}
## [1] 0.1053754
\end{verbatim}

\end{document}
